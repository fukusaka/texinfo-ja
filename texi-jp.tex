%% TeX macros to handle Japanese texinfo files for Egg
%% Modified by Satoru Tomura (tomura@etl.go.jp)
%% 92.7.8   modified for Mule Ver.0.9.5 by K.Handa <handa@etl.go.jp>
%%      To detect type of jTeX and its version, the method
%%      posted by Takafumi SAKURAI <sakurai@math.metro-u.ac.jp> is used.
%% 92.9.30  modified for Mule Ver.0.9.6 by K.Handa <handa@etl.go.jp>
%%      For unknown reason, \newif\ifNTTOLD should be before
%%      \ifNTT.
%% 93.4.29  modified for Mule Ver.0.9.7 by N.Hikichi <hikichi@sra.co.jp>
%% 95.10.6  modified for texinfo 2.145 by K.Handa <handa@etl.go.jp>
%% 95.10.13 modified by J.Sato <jun@svgw.rd.casio.co.jp>
%%      Support many Japanese oriented phrases (reference, etc)
%% 95.10.14 modified by K.Handa <handa@etl.go.jp>
%%      Bug for handling index fixed.
%% 96.1.16 modified by J.Sato <jun@svgw.rd.casio.co.jp>
%%      index with [] of @deffn.
%% 99.6.27 modified by Moimoi <fukusaka@xa2.so-net.ne.jp>
%%	for texinfo 1999-05-25.6

%% 92.7.8 by K.Handa
\newif\ifNTT
\ifx\gtfam\undefined
\NTTtrue
\else
\NTTfalse
\fi

\newif\ifNTTOLD
\ifNTT
\ifx\jendlinetype\undefined
\NTTOLDtrue
\else
\NTTOLDfalse
\fi
\fi
%% end of patch

%% TeX macros to handle Japanese texinfo files
%% 92/05/24 merged jtexinfo.tex (by H. Isozaki and N. Hikichi) into this
%% Created by Satoru Tomura (tomura@etl.go.jp)

\def\texinfoJPversion{2.145J.2}

\ifNTT
\message{texi-jp (NTT JTeX) package [Version \texinfoJPversion]:}
\else
\message{texi-jp (ASCII JTeX) package [Version \texinfoJPversion]:}
\fi
\message{}

%% 日本語フォントに関する互換性
\ifNTT
\kanjifiletype=20
\let\min=\dm\let\dg=\goth
\else
\let\dm=\min\let\goth=\dg
\fi

\setfont\textrm\rmshape{10}{1000}
\setfont\texttt\ttshape{10}{1000}
\setfont\textbf\bfshape{10}{1000}
\setfont\textit\itshape{10}{1000}
\setfont\textsl\slshape{10}{1000}
\setfont\textsf\sfshape{10}{1000}
\setfont\textsc\scshape{10}{1000}
\font\texti=cmmi10

%% 日本語フォントの定義
%% Fonts for text
\ifNTT
\ifNTTOLD                       % 92.7.8 by K.Handa
\jfont\textdm=dm10 scaled {\magstephalf}
\jfont\textdg=dg10 scaled {\magstephalf}
\jfont\shortcontdm=dm10 scaled {\magstep1}
\jfont\shortcontdg=dg10 scaled {\magsteph1}
\else
%\jfont\textdm=dm10 scaled \magstephalf
%\jfont\textdg=dg10 scaled \magstephalf
%\jfont\shortcontdm=dm10 scaled \magstephalf
\jfont\textdm=dm10
\jfont\textdg=dg10
\jfont\shortcontdm=dm12
\jfont\shortcontdg=dg12
\fi
\else
%\font\textdm=min10 scaled \magstephalf
%\font\textdg=goth10 scaled \magstephalf
%\jfont\shortcontdm=min10 scaled \magstephalf
\font\textdm=min10
\font\textdg=goth10
\jfont\shortcontdm=min10 scaled \magstep1
\jfont\shortcontdg=goth10 scaled \magstep1
\fi

%% Fonts for title
\ifNTT
\jfont\titledm=dg12 scaled \magstep3
\else
%\font\titledm=goth12 scaled \magstep3
\font\titledm=goth10 scaled \magstep4
\fi

%% Fonts for indics and small examples
\ifNTT
\jfont\inddm=dm9
\jfont\inddg=dm9
\else
\font\inddm=min9
\font\inddg=min9
\fi

%% Fonts for headings
\ifNTT
\jfont\chapdm=dg12 scaled \magstep2
\else
%\font\chapdm=min12 scaled \magstep2
\font\chapdm=goth10 scaled \magstep3
\fi
\let\chapdg=\chapdm

\ifNTT
\jfont\secdm=dg12 scaled \magstep1
\jfont\secdg=dg12 scaled \magstep1
\else
%\font\secdm=min12 scaled \magstep1
\font\secdm=goth10 scaled \magstep2
%\font\secdg=goth12 scaled \magstep1
\font\secdg=goth10 scaled \magstep2
\fi

\ifNTT
\ifNTTOLD                       % 92.7.8 by K.Handa
\jfont\ssecdm=dg12 scaled {\magstephalf}
\jfont\ssecdg=dg12 scaled {\magstephalf}
\else
\jfont\ssecdm=dg12 scaled \magstephalf
\jfont\ssecdg=dg12 scaled \magstephalf
\fi
\else
%\font\ssecdm=goth12 scaled \magstephalf
\font\ssecdm=goth10 at 13pt
%\font\ssecdg=goth12 scaled \magstephalf
\font\ssecdg=goth10 at 13pt
\fi

% 95.11.2 by K.Handa
% Reduce Overfull/Underfull \hbox by relaxing these glues.
\ifNTT
\jintercharskip=0pt plus 0.5pt minus -0.2pt
\jasciikanjiskip=2.28854pt plus 0.5pt minus -0.2pt
\fi

%% Re-definitions
\let\origrm=\rm
\def\rm{\origrm\tendm}
\let\origbf=\bf
\def\bf{\origbf\tendg}
\let\origsl=\sl
\def\sl{\origsl\tendg}
\let\origauthorrm=\authorrm
\def\authorrm{\origauthorrm\secdm}


\let\origtextfonts=\textfonts
\def\textfonts{\let\tendm=\textdm\let\tendg=\textdg\origtextfonts\textrm\textdm
}
\let\origtitlefonts=\titlefonts
\def\titlefonts{\let\tendm=\titledm\let\tendg=\titledg\origtitlefonts}
\let\origchapfonts=\chapfonts
\def\chapfonts{\let\tendm=\chapdm\let\tendg=\chapdg\origchapfonts}
\let\origsecfonts=\secfonts
\def\secfonts{\let\tendm=\secdm\let\tendg=\secdg\origsecfonts}
\let\origsubsecfonts=\subsecfonts
\def\subsecfonts{\let\tendm=\ssecdm\let\tendg=\ssecdg\origsubsecfonts}
\let\origindexfonts=\indexfonts
\def\indexfonts{\let\tendm=\inddm\let\tendg=\inddg\origindexfonts}

% 95.11.2 by K.Handa
% Not to cause Underfull \hbox while skipping the block:
%       @ifset jp-draft ... @end ifset
\let\origignoremorecommands=\ignoremorecommands
\def\ignoremorecommands{
\origignoremorecommands
\let\result=\relax\let\expansion=\relax\let\print=\relax\let\equiv=\relax
}

% 95.11.2 by K.Handa
% The orignal codes have tailing \null, but JTeX doesn't
% insert \jasciikanjiskip in that case.  So the tailing
% \null is delted.
\def\t#1{{\tt \nohyphenation \rawbackslash \frenchspacing #1}}
\let\ttfont = \t
\def\samp #1{`\tclose{#1}'}
\let\file=\samp
\let\option=\samp
\def\key #1{{\tt \nohyphenation \uppercase{#1}}}
\def\tclose#1{%
  {%
    % Change normal interword space to be same as for the current font.
    \spaceskip = \fontdimen2\font
    %
    % Switch to typewriter.
    \tt
    %
    % But `\ ' produces the large typewriter interword space.
    \def\ {{\spaceskip = 0pt{} }}%
    %
    % Turn off hyphenation.
    \nohyphenation
    %
    \rawbackslash
    \frenchspacing
    #1%
  }%
}

% The same can be said for the trailing \/.
\def\smartitalicx{\ifx\next,\else\ifx\next-\else\/\fi\fi}

\def\initial#1{{%
  % Some minor font changes for the special characters.
  \let\tentt=\sectt \let\tt=\sectt \let\sf=\sectt
  %
  % Remove any glue we may have, we'll be inserting our own.
  \removelastskip
  %
  % We like breaks before the index initials, so insert a bonus.
  \penalty -300
  %
  % Typeset the initial.  Making this add up to a whole number of
  % baselineskips increases the chance of the dots lining up from column
  % to column.  It still won't often be perfect, because of the stretch
  % we need before each entry, but it's better.
  %
  % No shrink because it confuses \balancecolumns.
  \vskip 1.67\baselineskip plus .5\baselineskip
  \leftline{\secbf\secdg #1}%
  \vskip .33\baselineskip plus .1\baselineskip
  %
  % Do our best not to break after the initial.
  \nobreak
}}

% 非互換な日本語化部分

% Set up fixed words for Japanese.
\gdef\putwordChapter{章}%
\gdef\putwordSee{を参照してください}%
\gdef\putwordsee{参照}%
\gdef\putwordfile{ファイル}%
\gdef\putwordpage{p.}%
\gdef\putwordsection{節}%
\gdef\putwordSection{節}%
\gdef\putwordTOC{目次}%
\gdef\putwordShortTOC{簡略目次}%
\gdef\putwordAppendix{付録}%

\gdef\putwordDefmac{変数}
\gdef\putwordDefspec{Special Form}
\gdef\putwordDefvar{変数}
\gdef\putwordDefopt{User Option}
\gdef\putwordDeftypevar{変数}
\gdef\putwordDeffunc{関数}
\gdef\putwordDeftypefun{関数}


\def\today{\number\year 年 \number\month 月 \number\day 日}

\setbox0 = \hbox{\shortcontrm\shortcontdm \putwordAppendix }
\shortappendixwidth = \wd0

% And just the chapters.
\def\summarycontents{%
   \startcontents{\putwordShortTOC}%
      %
      \let\chapentry = \shortchapentry
      \let\unnumbchapentry = \shortunnumberedentry
      % We want a true roman here for the page numbers.
      \secfonts
      \def\rm{\shortcontrm\shortcontdm}
      \def\bf{\shortcontbf\shortcontdg}
      \def\sl{\shortcontsl\shortcontdg}
      \rm
      \hyphenpenalty = 10000
      \advance\baselineskip by 1pt % Open it up a little.
      \def\secentry ##1##2##3##4{}
      \def\unnumbsecentry ##1##2{}
      \def\subsecentry ##1##2##3##4##5{}
      \def\unnumbsubsecentry ##1##2{}
      \def\subsubsecentry ##1##2##3##4##5##6{}
      \def\unnumbsubsubsecentry ##1##2{}
      \openin 1 \jobname.toc
      \ifeof 1 \else
        \closein 1
        \input \jobname.toc
      \fi
     \vfill \eject
     \contentsalignmacro % in case @setchapternewpage odd is in effect
   \endgroup
   \lastnegativepageno = \pageno
   \pageno = \savepageno
}
\let\shortcontents = \summarycontents

\def\inforefzzz #1,#2,#3,#4**{\putwordInfo{}\putwordfile{} \file{\ignorespaces #3{}},  ノード\samp{\ignorespaces#1{}}\putwordSee{}}

\def\chapterzzz #1{%
\secno=0 \subsecno=0 \subsubsecno=0
\global\advance \chapno by 1 \message{第\the\chapno\putwordChapter}%
\chapmacro {#1}{\the\chapno}%
\gdef\thissection{#1}%
\gdef\thischaptername{#1}%
% We don't substitute the actual chapter name into \thischapter
% because we don't want its macros evaluated now.
\xdef\thischapter{第\the\chapno\putwordChapter{}: \noexpand\thischaptername}%
\toks0 = {#1}%
\edef\temp{\noexpand\writetocentry{\realbackslash chapentry{\the\toks0}%
                                  {\the\chapno}}}%
\temp
\donoderef
\global\let\section = \numberedsec
\global\let\subsection = \numberedsubsec
\global\let\subsubsection = \numberedsubsubsec
}

\def\defthissection#1{%
{\def\code##1{'code {##1}}\def\file##1{'file {##1}}\xdef\thissection{#1}}}

\def\numberedsubseczzz #1{%
\defthissection{#1}\subsubsecno=0 \global\advance \subsecno by 1 %
\subsecheading {#1}{\the\chapno}{\the\secno}{\the\subsecno}%
\toks0 = {#1}%
\edef\temp{\noexpand\writetocentry{\realbackslash subsecentry{\the\toks0}%
                                    {\the\chapno}{\the\secno}{\the\subsecno}}}%
\temp
\donoderef
\nobreak
}

\def\numberedsubsubseczzz #1{%
\defthissection{#1}\global\advance \subsubsecno by 1 %
\subsubsecheading {#1}
  {\the\chapno}{\the\secno}{\the\subsecno}{\the\subsubsecno}%
\toks0 = {#1}%
\edef\temp{\noexpand\writetocentry{\realbackslash subsubsecentry{\the\toks0}%
  {\the\chapno}{\the\secno}{\the\subsecno}{\the\subsubsecno}}}%
\temp
\donoderef
\nobreak
}

\def\pxref#1{\xrefX[#1,,,,,,,]\putwordsee{}}
\def\xref#1{\xrefX[#1,,,,,,,]\putwordSee{}}

\def\xrefX[#1,#2,#3,#4,#5,#6]{\begingroup
  \def\printedmanual{\ignorespaces #5}%
  \def\printednodename{\ignorespaces #3}%
  \setbox1=\hbox{\printedmanual}%
  \setbox0=\hbox{\printednodename}%
  \ifdim \wd0 = 0pt
    % No printed node name was explicitly given.
    \expandafter\ifx\csname SETxref-automatic-section-title\endcsname\relax
      % Use the node name inside the square brackets.
      \def\printednodename{\ignorespaces #1}%
    \else
      % Use the actual chapter/section title appear inside
      % the square brackets.  Use the real section title if we have it.
      \ifdim \wd1 > 0pt
        % It is in another manual, so we don't have it.
        \def\printednodename{\ignorespaces #1}%
      \else
        \ifhavexrefs
          % We know the real title if we have the xref values.
          \def\printednodename{\refx{#1-title}{}}%
        \else
          % Otherwise just copy the Info node name.
          \def\printednodename{\ignorespaces #1}%
        \fi%
      \fi
    \fi
  \fi
  %
  % If we use \unhbox0 and \unhbox1 to print the node names, TeX does not
  % insert empty discretionaries after hyphens, which means that it will
  % not find a line break at a hyphen in a node names.  Since some manuals
  % are best written with fairly long node names, containing hyphens, this
  % is a loss.  Therefore, we give the text of the node name again, so it
  % is as if TeX is seeing it for the first time.
  \ifpdf
    \leavevmode
    \getfilename{#4}%
    \ifnum\filenamelength>0
      \pdfannotlink attr{/Border [0 0 0]}%
        goto file{\the\filename.pdf} name{#1@}%
    \else
      \pdfannotlink attr{/Border [0 0 0]}%
        goto name{#1@}%
    \fi
    \BlueGreen
  \fi
  %
  \ifdim \wd1 > 0pt
    \cite{\printedmanual}の``\printednodename''\putwordsection{}%
  \else
    % _ (for example) has to be the character _ for the purposes of the
    % control sequence corresponding to the node, but it has to expand
    % into the usual \leavevmode...\vrule stuff for purposes of
    % printing. So we \turnoffactive for the \refx-snt, back on for the
    % printing, back off for the \refx-pg.
    {\normalturnoffactive
     % Only output a following space if the -snt ref is nonempty; for
     % @unnumbered and @anchor, it won't be.
     \setbox2 = \hbox{\ignorespaces \refx{#1-snt}{}}%
     \ifdim \wd2 > 0pt \refx{#1-snt}\fi
    }「\printednodename 」%
    % page 3
%    \turnoffactive \putwordpage\tie\refx{#1-pg}{}%
    \turnoffactive \refx{#1-pg}{}%
  \fi
  \ifpdf \Black\pdfendlink \fi
\endgroup}

% added by Moimoi
% p.と数字な間隔を少なくするため。
\def\Ypagenumber{\putwordpage\folio}

\def\Ysectionnumberandtype{%
\ifnum\secno=0 第\the\chapno\putwordChapter%
\else \ifnum \subsecno=0 \the\chapno.\the\secno\putwordSection%
\else \ifnum \subsubsecno=0 %
\xreftie\the\chapno.\the\secno.\the\subsecno\putwordSection%
\else %
\xreftie\the\chapno.\the\secno.\the\subsecno.\the\subsubsecno\putwordSection%
\fi \fi \fi }

% added by Moimoi
\appendixno = 0
\def\appendixletter

\def\Yappendixletterandtype{%
\ifnum\secno=0 \putwordAppendix\appendixletter%
\else \ifnum \subsecno=0 \appendixletter.\the\secno\putwordSection
%
\else \ifnum \subsubsecno=0 %
\appendixletter.\the\secno.\the\subsecno\putwordSection %
\else %
\appendixletter.\the\secno.\the\subsecno.\the\subsubsecno\putwordSection %
\fi \fi \fi }

%\def\Yappendixletterandtype{%
%\ifnum\secno=0 \putwordAppendix\xreftie'char\the\appendixno{}%
%\else \ifnum \subsecno=0 \xreftie'char\the\appendixno.\the\secno\putwordSection
%%
%\else \ifnum \subsubsecno=0 %
%\xreftie'char\the\appendixno.\the\secno.\the\subsecno\putwordSection %
%\else %
%\xreftie'char\the\appendixno.\the\secno.\the\subsecno.\the\subsubsecno\putwordS
%ection %
%\fi \fi \fi }

% @dfn
\def\doublebracket#1{『#1』}
\let\dfn=\doublebracket

%
% A4 size(Japanese) define, top margin = 20, bottom margin = 21,
%  left margin = 30, right margin = 15
%
\def\a4book{
\global\lispnarrowing = 0.3in
\global\baselineskip 12pt
\global\parskip 3pt plus 1pt

% for @cropmarks
%\global\hsize = 6.5in
% without @cropmarks
\global\hsize = 6.7in

\global\doublecolumnhsize=2.4in \global\doublecolumnvsize=15.0in
\global\vsize=9.8in
\global\tolerance=700
\global\hfuzz=1pt

\global\pagewidth=\hsize
\global\pageheight=\vsize
\global\font\ninett=cmtt9

\global\let\smalllisp=\smalllispx
\global\let\smallexample=\smalllispx
\global\def\Esmallexample{\Esmalllisp}

% for @cropmarks
%\global\voffset = -1.0in
%\global\hoffset = -0.2in

% without @cropmarks
\global\voffset = 0.0in
%\global\hoffset = -1.0in
\global\hoffset = -0.2in
}

% @smallbook for B5
\def\smallbook{
\outerhsize=182mm
\outervsize=257mm
\hoffset=-0.3in
\voffset=-0.3in

% These values for secheadingskip and subsecheadingskip are
% experiments.  RJC 7 Aug 1992
\global\secheadingskip = 17pt plus 6pt minus 3pt
\global\subsecheadingskip = 14pt plus 6pt minus 3pt

\global\lispnarrowing = 0.3in
\setleading{14pt}
\advance\topskip by -7mm
\global\parskip 3pt plus 1pt
\global\hsize = 5.5in
\global\vsize=8.25in
\global\tolerance=700
\global\hfuzz=1pt
\global\contentsrightmargin=0pt
\global\deftypemargin=0pt
\global\defbodyindent=.5cm

\global\pagewidth=\hsize
\global\pageheight=\vsize

\global\let\smalllisp=\smalllispx
\global\let\smallexample=\smalllispx
\global\def\Esmallexample{\Esmalllisp}
}

\newdimen\defaultparindent \defaultparindent = 1zw
\parindent = \defaultparindent

\def\includezzz#1{\endgroup\begingroup
  \openin 1 #1
  \ifeof 1\message{File #1 does not exist.}\else \closein 1
  % Read the included file in a group so nested @include's work.
  \def\thisfile{#1}%
  \input\thisfile
  \fi
\endgroup}


% added by J.Sato
% LaTeXのlinebreakの真似
% 半田さんより
\def\linebreak{\unskip\penalty -10000}

% インデックスにソートキーを[]で指定する.
\def\Jempty{}
\def\singleindexer #1{\singleindexerB#1[]\singleindexerA}
% 93.4.29 by N.Hikichi
%\def\singleindexerB#1[#2]{\edef\Jone{#1}\edef\Jtwo{#2}%
\def\singleindexerB#1[#2]{\def\Jone{#1}\def\Jtwo{#2}%
\ifx\Jempty\Jtwo\let\Jnext=\relax\let\singleindexerA=\singleindexerD%
\else\let\Jnext=\singleindexerC\let\singleindexerA=\singleindexerE\fi%
\Jnext}
\def\singleindexerC#1[]{}
\def\singleindexerD{\doind{\indexname}{\Jone}}
\def\singleindexerE{\Jdoind{\indexname}{\Jone}{\Jtwo}}


%% Most index entries go through here, but \dosubind is the general case.
%%
%\def\doind#1#2{\dosubind{#1}{#2}\empty}
%
%% Workhorse for all \fooindexes.
%% #1 is name of index, #2 is stuff to put there, #3 is subentry --
%% \empty if called from \doind, as we usually are.  The main exception
%% is with defuns, which call us directly.
%%
%\def\dosubind#1#2#3{%
\def\Jdoind#1#2#3{
  % Put the index entry in the margin if desired.
  \ifx\SETmarginindex\relax\else
    \insert\margin{\hbox{\vrule height8pt depth3pt width0pt #2}}%
  \fi
  {%
    \count255=\lastpenalty
    {%
      \indexdummies % Must do this here, since \bf, etc expand at this stage
      \escapechar=`\\
      {%
        \let\folio = 0% We will expand all macros now EXCEPT \folio.
        \def\rawbackslashxx{\indexbackslash}% \indexbackslash isn't defined now
        % so it will be output as is; and it will print as backslash.
        %
%	 \def\thirdarg{#3}%
%	 %
%	 % If third arg is present, precede it with space in sort key.
%	 \ifx\thirdarg\emptymacro
%	   \let\subentry = \empty
%	 \else
%	   \def\subentry{ #3}%
%	 \fi
	%
        % First process the index entry with all font commands turned
        % off to get the string to sort by.
%        {\indexnofonts \xdef\indexsorttmp{#2\subentry}}%
	{\indexnofonts \xdef\indexsorttmp{#3}}%
        %
        % Now the real index entry with the fonts.
%	 \toks0 = {#2}%
%	 %
%	 % If third (subentry) arg is present, add it to the index
%	 % string.  And include a space.
%	 \ifx\thirdarg\emptymacro \else
%	   \toks0 = \expandafter{\the\toks0 \space #3}%
%	 \fi
        %
        % Set up the complete index entry, with both the sort key
        % and the original text, including any font commands.  We write
        % three arguments to \entry to the .?? file, texindex reduces to
        % two when writing the .??s sorted result.
        \edef\temp{%
          \write\csname#1indfile\endcsname{%
%            \realbackslash entry{\indexsorttmp}{\folio}{\the\toks0}}%
            \realbackslash entry{\indexsorttmp}{\folio}{#2}}%
        }%
        %
        % If a skip is the last thing on the list now, preserve it
        % by backing up by \lastskip, doing the \write, then inserting
        % the skip again.  Otherwise, the whatsit generated by the
        % \write will make \lastskip zero.  The result is that sequences
        % like this:
        % @end defun
        % @tindex whatever
        % @defun ...
        % will have extra space inserted, because the \medbreak in the
        % start of the @defun won't see the skip inserted by the @end of
        % the previous defun.
        %
        % But don't do any of this if we're not in vertical mode.  We
        % don't want to do a \vskip and prematurely end a paragraph.
        %
        % Avoid page breaks due to these extra skips, too.
        %
        \iflinks
          \ifvmode
            \skip0 = \lastskip
            \ifdim\lastskip = 0pt \else \nobreak\vskip-\lastskip \fi
          \fi
          %
          \temp % do the write
          %
          %
          \ifvmode \ifdim\skip0 = 0pt \else \nobreak\vskip\skip0 \fi \fi
        \fi
      }%
    }%
    \penalty\count255
  }%
}

% ???? 動かない、、、。
%
% added by J.Sato
% @deffnで読みを指定する
%{\activeparens%
%\def\deffnheader #1#2{\deffnheaderB{#1}#2[]\deffnheaderA}
%\def\deffnheaderB#1#2[#3]{\def\Jone{#1}\def\Jtwo{#2}\def\Jthree{#3}%
%\ifx\Jempty\Jthree%
%\let\Jnext=\relax%
%\let\deffnheaderA=\deffnheaderD%
%\else%
%\let\Jnext=\deffnheaderC%
%\let\deffnheaderA=\deffnheaderE%
%\fi\Jnext}
%
%\def\deffnheaderC#1[]{}
%\def\deffnheaderD#1{\doind {fn}{\code{\Jtwo}}%
%\begingroup\defname {\Jtwo}{\Jone}\defunargs{#1}\endgroup %
%\catcode 61=\other % Turn off change made in \defparsebody
%}
%\def\deffnheaderE#1{\Jdoind {fn}{\code{\Jtwo}}{\Jthree}
%\begingroup\defname {\Jtwo}{\Jone}\defunargs{#1}\endgroup %
%\catcode 61=\other
%}
%}

