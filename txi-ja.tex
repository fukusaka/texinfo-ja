%% TeX macros to handle Japanese texinfo files
%%
%% 92.7.8   modified for Mule Ver.0.9.5 by K.Handa <handa@etl.go.jp>
%%      To detect type of jTeX and its version, the method
%%      posted by Takafumi SAKURAI <sakurai@math.metro-u.ac.jp> is used.
%% 92.9.30  modified for Mule Ver.0.9.6 by K.Handa <handa@etl.go.jp>
%%      For unknown reason, \newif\ifNTTOLD should be before
%%      \ifNTT.
%% 93.4.29  modified for Mule Ver.0.9.7 by N.Hikichi <hikichi@sra.co.jp>
%% 95.10.6  modified for texinfo 2.145 by K.Handa <handa@etl.go.jp>
%% 95.10.13 modified by J.Sato <jun@svgw.rd.casio.co.jp>
%%      Support many Japanese oriented phrases (reference, etc)
%% 95.10.14 modified by K.Handa <handa@etl.go.jp>
%%      Bug for handling index fixed.
%% 96.1.16 modified by J.Sato <jun@svgw.rd.casio.co.jp>
%%      index with [] of @deffn.
%% 99.6.27 modified by Moimoi <fukusaka@xa2.so-net.ne.jp>
%%	for texinfo 1999-05-25.6
%% 2000.2.23 modified by Moimoi <fukusaka@xa2.so-net.ne.jp>
%%	for texinfo.tex 1999-09-25.10
%% 2000.4.11 modified by Moimoi <fukusaka@xa2.so-net.ne.jp>
%%      for texinfo.tex 1999-09-25.10
%%      fixed for jTeX/pTeX/MulTeX
%% 2010.06.02 modified by Shoichi Fukusaka <fukusaka@xa2.so-net.ne.jp>
%%      for texinfo.tex 2005-01-30.17
%%      omitted support for old NTT-jTeX/MulTex
%%      improve macros
%%%%%%%%%%%%%%%%%%%%%%%%%%%%%%%%%%%%%%%%%%%%%%%%%%%%%%%%%%%%%%%%
%% 92.7.8 by K.Handa
\newif\ifNTT
\ifx\gtfam\undefined
\NTTtrue
\else
\NTTfalse
\fi

%% TeX macros to handle Japanese texinfo files
%% 92/05/24 merged jtexinfo.tex (by H. Isozaki and N. Hikichi) into this
%% Created by Satoru Tomura (tomura@etl.go.jp)

\def\texinfoJPversion{5}

\ifNTT
\message{txi-ja (NTT JTeX) package [Version \texinfoJPversion]:}
\else
\message{txi-ja (ASCII pTeX) package [Version \texinfoJPversion]:}
\fi
\message{}

%%%%%%%%%%%%%%%%%%%%%%%%%%%%%%%%%%%%%%%%%%%%%%%%%%%%%%%%%%%%%%%%
% Set up fixed words for Japanese.
\gdef\putwordAppendix{付録}
\gdef\putwordChapter{章}
\gdef\putwordfile{ファイル}
\gdef\putwordin{in}
\gdef\putwordIndexIsEmpty{(索引が空です)}
\gdef\putwordIndexNonexistent{(索引がありません)}
\gdef\putwordInfo{Info}
\gdef\putwordInstanceVariableof{Instance Variable of}
\gdef\putwordMethodon{Method on}
\gdef\putwordNoTitle{無タイトル}
\gdef\putwordof{of}
\gdef\putwordon{on}
\gdef\putwordpage{p.\gobble}
\gdef\putwordsection{節}
\gdef\putwordSection{節}
\gdef\putwordsee{参照}
\gdef\putwordSee{を参照してください}
\gdef\putwordShortTOC{簡略目次}
\gdef\putwordTOC{目次}
%
\gdef\putwordMJan{1月}
\gdef\putwordMFeb{2月}
\gdef\putwordMMar{3月}
\gdef\putwordMApr{4月}
\gdef\putwordMMay{5月}
\gdef\putwordMJun{6月}
\gdef\putwordMJul{7月}
\gdef\putwordMAug{8月}
\gdef\putwordMSep{9月}
\gdef\putwordMOct{10月}
\gdef\putwordMNov{11月}
\gdef\putwordMDec{12月}
%
\gdef\putwordDefmac{マクロ}
\gdef\putwordDefspec{Special Form}
\gdef\putwordDefvar{変数}
\gdef\putwordDefopt{オプション}
\gdef\putwordDeffunc{関数}

\def\today{\number\year 年 \number\month 月 \number\day 日}

%%%%%%%%%%%%%%%%%%%%%%%%%%%%%%%%%%%%%%%%%%%%%%%%%%%%%%%%%%%%%%%%
% 日本語組の段落インデントのデフォルトを設定
\global\defaultparindent = 1em
\global\def\suppressfirstparagraphindent{\relax}

% 日本人好きのギッシリ詰まった紙
\global\def\afourpaperj{{\globaldefs = 1
  \parskip = 3pt plus 2pt minus 1pt
  \textleading = 12pt
  %
  \internalpagesizes{248mm}{170mm}%
                    {\voffset}{-6mm}%
                    {\bindingoffset}{8mm}%
                    {297mm}{210mm}%
  %
  \tolerance = 750
  \hfuzz = 1pt
  \contentsrightmargin = 0pt
  \defbodyindent = 5mm
}}

%%%%%%%%%%%%%%%%%%%%%%%%%%%%%%%%%%%%%%%%%%%%%%%%%%%%%%%%%%%%%%%%
%% Utility routines.

\def\csdef#1{\expandafter\def\csname#1\endcsname}

\gdef\addcsbefore#1#2{%
  \cslet{orig#1}{#1}%
  \csdef{addcs#1}{#2}%
  \csdef{#1}{%
    \cslet{addcs}{addcs#1}%
    \futurelet\orig\addcsbeforeyyy\expandafter\empty\csname orig#1\endcsname
  }
}
\gdef\addcsbeforeyyy{\addcs\orig}

\gdef\addcsafter#1#2{%
  \cslet{orig#1}{#1}%
  \csdef{addcs#1}{#2}%
  \csdef{#1}{%
    \cslet{addcs}{addcs#1}%
    \futurelet\orig\addcsafteryyy\expandafter\empty\csname orig#1\endcsname
  }
}
\gdef\addcsafteryyy{\orig\addcs}

%%%%%%%%%%%%%%%%%%%%%%%%%%%%%%%%%%%%%%%%%%%%%%%%%%%%%%%%%%%%%%%%
%% 日本語フォント

\def\jafont{%
\ifNTT%
\let\next=\jTeXjafont%
\else%
\let\next=\pTeXjafont%
\fi%
\next}

\def\jTeXjafont#1#2#3#4{%
\def\tempa{#2}
\def\tempb{mc}
\ifx\tempa\tempb% mc
\expandafter\gjfont\csname#1\endcsname=dm#3 scaled #4%
\else% gt
\expandafter\gjfont\csname#1\endcsname=dg#3 scaled #4%
\fi%
}

\def\pTeXjafont#1#2#3#4{%
\def\tempa{#2}
\def\tempb{mc}
\ifx\tempa\tempb% mc
\global\expandafter\font\csname#1\endcsname=min#3 scaled #4%
\else% gt
\global\expandafter\font\csname#1\endcsname=goth#3 scaled #4%
\fi
}

%
% 日本語フォントの定義
%

% Text fonts (11.2pt, magstep1).
\jafont{textmc}{mc}{10}{\mainmagstep}
\jafont{textgt}{gt}{10}{\mainmagstep}

% Fonts for indices, footnotes, small examples (9pt).
\jafont{smallmc}{mc}{9}{1000}
\jafont{smallgt}{gt}{9}{1000}

% Fonts for small examples (8pt).
\jafont{smallermc}{mc}{8}{1000}
\jafont{smallergt}{gt}{8}{1000}

% Fonts for title page (20.4pt):
\jafont{titlemc}{mc}{10}{\magstep4}

% Chapter (and unnumbered) fonts (17.28pt).
\jafont{chapmc}{mc}{10}{\magstep3}
\jafont{chapgt}{gt}{10}{\magstep3}

% Section fonts (14.4pt).
\jafont{secmc}{mc}{10}{\magstep2}
\jafont{secgt}{gt}{10}{\magstep2}

% Subsection fonts (13.15pt).
\jafont{ssecmc}{mc}{10}{1315}
\jafont{ssecgt}{gt}{10}{1315}

% Reduced fonts for @acro in text (10pt).
\jafont{reducedmc}{mc}{10}{1000}
\jafont{reducedgt}{gt}{10}{1000}

% Fonts for short table of contents. (12pt)
\jafont{shortcontmc}{mc}{10}{\magstep1}
\jafont{shortcontgt}{gt}{10}{\magstep1}

% 95.11.2 by K.Handa
% Reduce Overfull/Underfull \hbox by relaxing these glues.
\ifNTT
\global\jintercharskip=0pt plus 0.5pt minus -0.2pt
\global\jasciikanjiskip=2.28854pt plus 0.5pt minus -0.2pt
\fi

%%%%%%%%%%%%%%%%%%%%%%%%%%%%%%%%%

\def\syncjfont#1#2{\addcsafter{#1}{\csname #2\endcsname}}
\def\addjfonts#1#2{\addcsafter{#1fonts}{\cslet{tenmc}{#2mc}\cslet{tengt}{#2gt}}}

%%%%%%%%%%%%%%%%%%%%%%%%%%%%%%%%%
\globaldefs = 1

\syncjfont{rm}{tenmc}
\syncjfont{bf}{tengt}
\syncjfont{sl}{tengt}
\syncjfont{authorrm}{secmc}
\syncjfont{authortt}{secgt}

\addjfonts{text}{text}
\addjfonts{title}{title}
\addjfonts{chap}{chap}
\addjfonts{sec}{sec}
\addjfonts{subsec}{ssec}
\addjfonts{small}{small}
\addjfonts{smaller}{smaller}
\addjfonts{reduced}{reduced}

\let\subsubsecfonts = \subsecfonts
\let\smallexamplefonts = \smallfonts

\textfonts \rm

\globaldefs = 0

%%%%%%%%%%%%%%%%%%%%%%%%%%%%%%%%%%%%%%%%%%%%%%%%%%%%%%%%%%%%%%%%%%
% 日本語表記の改善
%  1. ヘッドの章の表記
%  2. See 訳語に合わせて動詞の後置
%  3. クロスレファレンスの表記

%%%%%%%%%%%%%%%%%%%%%%%%%%%%%%%%%
%  3. クロスレファレンスの表記
%  1. ヘッドの章の表記
\global\def\thischapterspace{\hskip \SETthischapterspace em}
\set thischapterspace 1

\global\def\chapmacro#1#2#3{%
  \pchapsepmacro
  {%
    \chapfonts \rm
    %
    % Have to define \thissection before calling \donoderef, because the
    % xref code eventually uses it.  On the other hand, it has to be called
    % after \pchapsepmacro, or the headline will change too soon.
    \gdef\thissection{#1}%
    \gdef\thischaptername{#1}%
    %
    % Only insert the separating space if we have a chapter/appendix
    % number, and don't print the unnumbered ``number''.
    \def\temptype{#2}%
    \ifx\temptype\Ynothingkeyword
      \setbox0 = \hbox{}%
      \def\toctype{unnchap}%
      \gdef\thischapter{#1}%
    \else\ifx\temptype\Yomitfromtockeyword
      \setbox0 = \hbox{}% contents like unnumbered, but no toc entry
      \def\toctype{omit}%
      \gdef\thischapter{}%
    \else\ifx\temptype\Yappendixkeyword
      \setbox0 = \hbox{\putwordAppendix{} #3\enspace}%
      \def\toctype{app}%
      % We don't substitute the actual chapter name into \thischapter
      % because we don't want its macros evaluated now.  And we don't
      % use \thissection because that changes with each section.
      %
      %\xdef\thischapter{\putwordAppendix{} \appendixletter:
      %                  \noexpand\thischaptername}%
      \xdef\thischapter{\putwordAppendix{}\appendixletter\thischapterspace
                        \noexpand\thischaptername}%
    \else
      \setbox0 = \hbox{#3\enspace}%
      \def\toctype{numchap}%
      %\xdef\thischapter{\putwordChapter{} \the\chapno:
      %                  \noexpand\thischaptername}%
      \xdef\thischapter{第\the\chapno\putwordChapter{}\thischapterspace
                        \noexpand\thischaptername}%
    \fi\fi\fi
    %
    % Write the toc entry for this chapter.  Must come before the
    % \donoderef, because we include the current node name in the toc
    % entry, and \donoderef resets it to empty.
    \writetocentry{\toctype}{#1}{#3}%
    %
    % For pdftex, we have to write out the node definition (aka, make
    % the pdfdest) after any page break, but before the actual text has
    % been typeset.  If the destination for the pdf outline is after the
    % text, then jumping from the outline may wind up with the text not
    % being visible, for instance under high magnification.
    \donoderef{#2}%
    %
    % Typeset the actual heading.
    \vbox{\hyphenpenalty=10000 \tolerance=5000 \parindent=0pt \raggedright
          \hangindent=\wd0 \centerparametersmaybe
          \unhbox0 #1\par}%
  }%
  \nobreak\bigskip % no page break after a chapter title
  \nobreak
}

\global\def\Ynumbered{%
  \ifnum\secno=0
    第\the\chapno\putwordChapter%
  \else \ifnum\subsecno=0
    \the\chapno.\the\secno\putwordSection%
  \else \ifnum\subsubsecno=0
    \the\chapno.\the\secno.\the\subsecno\putwordSection%
  \else
    \the\chapno.\the\secno.\the\subsecno.\the\subsubsecno\putwordSection%
  \fi\fi\fi
}

\global\def\Yappendix{%
  \ifnum\secno=0
     \putwordAppendix@tie @char\the\appendixno{}%
  \else \ifnum\subsecno=0
     @char\the\appendixno.\the\secno\putwordSection%
  \else \ifnum\subsubsecno=0
     @char\the\appendixno.\the\secno.\the\subsecno\putwordSection%
  \else
     @char\the\appendixno.\the\secno.\the\subsecno.\the\subsubsecno\putwordSection%
  \fi\fi\fi
}

%%%%%%%%%%%%%%%%%%%%%%%%%%%%%%%%%
%  2. See 訳語に合わせて動詞の後置

\global\def\inforefzzz #1,#2,#3,#4**{\putwordInfo{}\putwordfile{} \file{\ignorespaces #3{}},  ノード\samp{\ignorespaces#1{}}\putwordSee{}}

\global\def\pxref#1{\xrefX[#1,,,,,,,]\putwordsee{}}
\global\def\xref#1{\xrefX[#1,,,,,,,]\putwordSee{}}

%%%%%%%%%%%%%%%%%%%%%%%%%%%%%%%%%
%  3. クロスレファレンスの表記

\global\def\xrefX[#1,#2,#3,#4,#5,#6]{\begingroup
  \unsepspaces
  \def\printedmanual{\ignorespaces #5}%
  \def\printedrefname{\ignorespaces #3}%
  \setbox1=\hbox{\printedmanual\unskip}%
  \setbox0=\hbox{\printedrefname\unskip}%
  \ifdim \wd0 = 0pt
    % No printed node name was explicitly given.
    \expandafter\ifx\csname SETxref-automatic-section-title\endcsname\relax
      % Use the node name inside the square brackets.
      \def\printedrefname{\ignorespaces #1}%
    \else
      % Use the actual chapter/section title appear inside
      % the square brackets.  Use the real section title if we have it.
      \ifdim \wd1 > 0pt
        % It is in another manual, so we don't have it.
        \def\printedrefname{\ignorespaces #1}%
      \else
        \ifhavexrefs
          % We know the real title if we have the xref values.
          \def\printedrefname{\refx{#1-title}{}}%
        \else
          % Otherwise just copy the Info node name.
          \def\printedrefname{\ignorespaces #1}%
        \fi%
      \fi
    \fi
  \fi
  %
  % Make link in pdf output.
  \ifpdf
    \leavevmode
    \getfilename{#4}%
    {\turnoffactive \otherbackslash
     \ifnum\filenamelength>0
       \startlink attr{/Border [0 0 0]}%
         goto file{\the\filename.pdf} name{#1}%
     \else
       \startlink attr{/Border [0 0 0]}%
         goto name{\pdfmkpgn{#1}}%
     \fi
    }%
    \linkcolor
  \fi
  %
  % Float references are printed completely differently: "Figure 1.2"
  % instead of "[somenode], p.3".  We distinguish them by the
  % LABEL-title being set to a magic string.
  {%
    % Have to otherify everything special to allow the \csname to
    % include an _ in the xref name, etc.
    \indexnofonts
    \turnoffactive
    \otherbackslash
    \expandafter\global\expandafter\let\expandafter\Xthisreftitle
      \csname XR#1-title\endcsname
  }%
  \iffloat\Xthisreftitle
    % If the user specified the print name (third arg) to the ref,
    % print it instead of our usual "Figure 1.2".
    \ifdim\wd0 = 0pt
      \refx{#1-snt}%
    \else
      \printedrefname
    \fi
    %
    % if the user also gave the printed manual name (fifth arg), append
    % "in MANUALNAME".
    \ifdim \wd1 > 0pt
      \space \putwordin{} \cite{\printedmanual}%
    \fi
  \else
    % node/anchor (non-float) references.
    %
    % If we use \unhbox0 and \unhbox1 to print the node names, TeX does not
    % insert empty discretionaries after hyphens, which means that it will
    % not find a line break at a hyphen in a node names.  Since some manuals
    % are best written with fairly long node names, containing hyphens, this
    % is a loss.  Therefore, we give the text of the node name again, so it
    % is as if TeX is seeing it for the first time.
    \ifdim \wd1 > 0pt
       %\putwordsection{} ``\printedrefname'' \putwordin{} \cite{\printedmanual}%
       \cite{\printedmanual}の``\printednodename''\putwordsection{}%
    \else
      % _ (for example) has to be the character _ for the purposes of the
      % control sequence corresponding to the node, but it has to expand
      % into the usual \leavevmode...\vrule stuff for purposes of
      % printing. So we \turnoffactive for the \refx-snt, back on for the
      % printing, back off for the \refx-pg.
      {\turnoffactive \otherbackslash
       % Only output a following space if the -snt ref is nonempty; for
       % @unnumbered and @anchor, it won't be.
       \setbox2 = \hbox{\ignorespaces \refx{#1-snt}{}}%
       %\ifdim \wd2 > 0pt \refx{#1-snt}\space\fi
       \ifdim \wd2 > 0pt \refx{#1-snt}\fi
      }%
      % output the `[mynode]' via a macro so it can be overridden.
      \xrefprintnodename\printedrefname
      %
      % But we always want a comma and a space:
      %,\space
      %
      % output the `page 3'.
      \turnoffactive \otherbackslash \putwordpage\tie\refx{#1-pg}{}%
    \fi
  \fi
  \endlink
\endgroup}

\global\def\xrefprintnodename#1{「#1」}

%%%%%%%%%%%%%%%%%%%%%%%%%%%%%%%%%
% @dfn
\global\def\doublebracket#1{『#1』}
\global\let\dfn=\doublebracket

