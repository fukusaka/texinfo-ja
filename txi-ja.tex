%% TeX macros to handle Japanese texinfo files for Egg
%% Modified by Satoru Tomura (tomura@etl.go.jp)
%% 92.7.8   modified for Mule Ver.0.9.5 by K.Handa <handa@etl.go.jp>
%%      To detect type of jTeX and its version, the method
%%      posted by Takafumi SAKURAI <sakurai@math.metro-u.ac.jp> is used.
%% 92.9.30  modified for Mule Ver.0.9.6 by K.Handa <handa@etl.go.jp>
%%      For unknown reason, \newif\ifNTTOLD should be before
%%      \ifNTT.
%% 93.4.29  modified for Mule Ver.0.9.7 by N.Hikichi <hikichi@sra.co.jp>
%% 95.10.6  modified for texinfo 2.145 by K.Handa <handa@etl.go.jp>
%% 95.10.13 modified by J.Sato <jun@svgw.rd.casio.co.jp>
%%      Support many Japanese oriented phrases (reference, etc)
%% 95.10.14 modified by K.Handa <handa@etl.go.jp>
%%      Bug for handling index fixed.
%% 96.1.16 modified by J.Sato <jun@svgw.rd.casio.co.jp>
%%      index with [] of @deffn.
%% 99.6.27 modified by Moimoi <fukusaka@xa2.so-net.ne.jp>
%%	for texinfo 1999-05-25.6
%% 2000.2.23 modified by Moimoi <fukusaka@xa2.so-net.ne.jp>
%%	for texinfo.tex 1999-09-25.10
%% 2000.4.11 modified by Moimoi <fukusaka@xa2.so-net.ne.jp>
%%      for texinfo.tex 1999-09-25.10
%%      fixed for jTeX/pTeX/MulTeX
%%%%%%%%%%%%%%%%%%%%%%%%%%%%%%%%%%%%%%%%%%%%%%%%%%%%%%%%%%%%%%%%
%% 92.7.8 by K.Handa
\newif\ifNTT
\ifx\gtfam\undefined
\NTTtrue
\else
\NTTfalse
\fi

\newif\ifNTTOLD
\ifNTT
\ifx\jendlinetype\undefined
\NTTOLDtrue
\else
\NTTOLDfalse
\fi
\fi

\newif\ifMULTEX
\ifx\mlbaseversion\undefined
\MULTEXfalse
\else
\MULTEXtrue
\fi

%% TeX macros to handle Japanese texinfo files
%% 92/05/24 merged jtexinfo.tex (by H. Isozaki and N. Hikichi) into this
%% Created by Satoru Tomura (tomura@etl.go.jp)

\def\texinfoJPversion{2.145J.2+++}
%% これはどう変えるのがいいのかなぁ

\ifNTT
\ifMULTEX
\message{txi-ja (Multi-Lingual TeX) package [Version \texinfoJPversion]:}
\else
\message{txi-ja (NTT JTeX) package [Version \texinfoJPversion]:}
\fi
\else
\message{txi-ja (ASCII JTeX) package [Version \texinfoJPversion]:}
\fi
\message{}

%%%%%%%%%%%%%%%%%%%%%%%%%%%%%%%%%%%%%%%%%%%%%%%%%%%%%%%%%%%%%%%%
% Set up fixed words for Japanese.
\gdef\putwordAppendix{付録}
\gdef\putwordChapter{章}
\gdef\putwordfile{ファイル}
\gdef\putwordin{in}
\gdef\putwordIndexIsEmpty{(索引が空です)}
\gdef\putwordIndexNonexistent{(索引がありません)}
\gdef\putwordInfo{Info}
\gdef\putwordInstanceVariableof{Instance Variable of}
\gdef\putwordMethodon{Method on}
\gdef\putwordNoTitle{無タイトル}
\gdef\putwordof{of}
\gdef\putwordon{on}
\gdef\putwordpage{p.\gobble}
\gdef\putwordsection{節}
\gdef\putwordSection{節}
\gdef\putwordsee{参照}
\gdef\putwordSee{を参照してください}
\gdef\putwordShortTOC{簡略目次}
\gdef\putwordTOC{目次}
%
\global\newif\ifIGUMO\IGUMOfalse
\gdef\igumo{\IGUMOtrue}
\gdef\putwordMJan{睦月}
\gdef\putwordMFeb{如月}
\gdef\putwordMMar{弥生}
\gdef\putwordMApr{卯月}
\gdef\putwordMMay{皐月}
\gdef\putwordMJun{水無月}
\gdef\putwordMJul{文月}
\gdef\putwordMAug{葉月}
\gdef\putwordMSep{長月}
\gdef\putwordMOct{\ifIGUMO{神在月}\else{神無月}\fi}
\gdef\putwordMNov{霜月}
\gdef\putwordMDec{師走}
%
\gdef\putwordDefmac{マクロ}
\gdef\putwordDefspec{Special Form}
\gdef\putwordDefvar{変数}
\gdef\putwordDefopt{オプション}
\gdef\putwordDeftypevar{変数}
\gdef\putwordDeffunc{関数}
\gdef\putwordDeftypefun{関数}

\def\today{\number\year 年 \number\month 月 \number\day 日}

%%%%%%%%%%%%%%%%%%%%%%%%%%%%%%%%%%%%%%%%%%%%%%%%%%%%%%%%%%%%%%%%
%
% A4 size(Japanese) define, top margin = 20, bottom margin = 21,
%  left margin = 30, right margin = 15
%

% ???
%\global\def\a4book{
%\global\lispnarrowing = 0.3in
%\global\baselineskip 12pt
%\global\parskip 3pt plus 1pt
%
%% for @cropmarks
%%\global\hsize = 6.5in
%% without @cropmarks
%\global\hsize = 6.7in
%
%\global\doublecolumnhsize=2.4in \global\doublecolumnvsize=15.0in
%\global\vsize=9.8in
%\global\tolerance=700
%\global\hfuzz=1pt
%
%\global\pagewidth=\hsize
%\global\pageheight=\vsize
%\global\font\ninett=cmtt9
%
%\global\let\smalllisp=\smalllispx
%\global\let\smallexample=\smalllispx
%\global\def\Esmallexample{\Esmalllisp}
%
%% for @cropmarks
%%\global\voffset = -1.0in
%%\global\hoffset = -0.2in
%
%% without @cropmarks
%\global\voffset = 0.0in
%%\global\hoffset = -1.0in
%\global\hoffset = -0.2in
%}

% 日本人好きのギッシリ詰まった紙
% hack please !!
\global\def\afourbook{{\globaldefs = 1
  \setleading{12pt}%
  \parskip = 3pt plus 2pt minus 1pt
  %
  \internalpagesizes{248mm}{170mm}{0mm}{-5mm}{0mm}{8mm}%
  %
  \tolerance = 700
  \hfuzz = 1pt
}}

%% @smallbook for B5
%\global\def\smallbook{
%\outerhsize=182mm
%\outervsize=257mm
%\hoffset=-0.3in
%\voffset=-0.3in
%
%% These values for secheadingskip and subsecheadingskip are
%% experiments.  RJC 7 Aug 1992
%\global\secheadingskip = 17pt plus 6pt minus 3pt
%\global\subsecheadingskip = 14pt plus 6pt minus 3pt
%
%\global\lispnarrowing = 0.3in
%\setleading{14pt}
%\advance\topskip by -7mm
%\global\parskip 3pt plus 1pt
%\global\hsize = 5.5in
%\global\vsize=8.25in
%\global\tolerance=700
%\global\hfuzz=1pt
%\global\contentsrightmargin=0pt
%\global\deftypemargin=0pt
%\global\defbodyindent=.5cm
%
%\global\pagewidth=\hsize
%\global\pageheight=\vsize
%
%\global\let\smalllisp=\smalllispx
%\global\let\smallexample=\smalllispx
%\global\def\Esmallexample{\Esmalllisp}
%}

%%%%%%%%%%%%%%%%%%%%%%%%%%%%%%%%%%%%%%%%%%%%%%%%%%%%%%%%%%%%%%%%
%% 日本語フォントに関する互換性
%
% Debian/Linux でパッケージ化されている
%  NTT jTeX / ASCII pTeX / MulTeX(日本語限定) のみ考慮している。
% (他のサイトの日本語TeXも同様だと期待してる、、、)
%
% xdvi/jdvi2kps で使っている vf の種類は 5,6,7,8,9,10/min,goth 。
%

\ifNTT
%\global\kanjifiletype=20 % ??
\global\let\min=\dm\global\let\goth=\dg
\else
\global\let\dm=\min\global\let\dg=\goth
\fi

\def\uniJFont{%
\ifNTT%
\ifNTTOLD                       % 92.7.8 by K.Handa
\let\next=\jTeXoldJFont%
\else
\ifMULTEX
\let\next=\MulTeXJFont%
\else
\let\next=\jTeXJFont%
\fi
\fi
\else%
\let\next=\pTeXJFont%
\fi%
\next}

\def\jTeXoldJFont#1#2#3#4{%
\def\tempa{#2}
\def\tempb{dm}
\ifx\tempa\tempb% dm
\expandafter\gjfont\csname#1\endcsname=dm#3 scaled {#4}%
\else% dg
\expandafter\gjfont\csname#1\endcsname=dg#3 scaled {#4}%
\fi%
}

% backward compatibility for JTeX で \gjfont が無いとは MulTeX め。

\def\MulTeXJFont#1#2#3#4{%
\def\tempa{#2}
\def\tempb{dm}
{\globaldefs=1%
\ifx\tempa\tempb% dm
\expandafter\jfont\csname#1\endcsname=dm#3 scaled #4%
\else% dg
\expandafter\jfont\csname#1\endcsname=dg#3 scaled #4%
\fi}%
}

\def\jTeXJFont#1#2#3#4{%
\def\tempa{#2}
\def\tempb{dm}
\ifx\tempa\tempb% dm
\expandafter\gjfont\csname#1\endcsname=dm#3 scaled #4%
\else% dg
\expandafter\gjfont\csname#1\endcsname=dg#3 scaled #4%
\fi%
}

\def\pTeXJFont#1#2#3#4{%
\def\tempa{#2}
\def\tempb{dm}
\ifx\tempa\tempb% dm
\global\expandafter\font\csname#1\endcsname=min#3 scaled #4%
\else% dg
\global\expandafter\font\csname#1\endcsname=goth#3 scaled #4%
\fi
}

%
% 日本語フォントの定義
%

%% Fonts for text (10pt)
\uniJFont{textdm}{dm}{10}{1000}
\uniJFont{textdg}{dg}{10}{1000}

\global\setfont\textrm\rmshape{10}{1000}
\global\setfont\texttt\ttshape{10}{1000}
\global\setfont\textbf\bfshape{10}{1000}
\global\setfont\textit\itshape{10}{1000}
\global\setfont\textsl\slshape{10}{1000}
\global\setfont\textsf\sfshape{10}{1000}
\global\setfont\textsc\scshape{10}{1000}
\global\setfont\textttsl\ttslshape{10}{1000}
\global\font\texti=cmmi10
\global\font\textsy=cmsy10

%% Fonts for shortcontext (12pt)
\uniJFont{shortcontdm}{dm}{10}{\magstep1}
\uniJFont{shortcontdg}{dg}{10}{\magstep1}

%% Fonts for title (20.74pt)
\uniJFont{titledm}{dg}{10}{\magstep4}

%% Fonts for indics and small examples
\uniJFont{smalldm}{dm}{9}{1000}
\uniJFont{smalldg}{dg}{9}{1000}

%% Fonts for headings (17.28pt)
\uniJFont{chapdm}{dg}{10}{\magstep3}
\uniJFont{chapdg}{dg}{10}{\magstep3}

%% Fonts for sections (14.40pt)
\uniJFont{secdm}{dm}{10}{\magstep2}
\uniJFont{secdg}{dg}{10}{\magstep2}

%% Fonts for subsections (13.15pt)
\uniJFont{ssecdm}{dm}{10}{1315}
\uniJFont{ssecdg}{dg}{10}{1315}

% 95.11.2 by K.Handa
% Reduce Overfull/Underfull \hbox by relaxing these glues.
\ifNTT
\global\jintercharskip=0pt plus 0.5pt minus -0.2pt
\global\jasciikanjiskip=2.28854pt plus 0.5pt minus -0.2pt
\fi

%%%%

%% (^^;)
%\global\def\tendm{}
%\global\def\tendg{}

%% Re-definitions
\gdef\addjfont#1#2{%
\cslet{orig#1}{#1}%
\expandafter\def\csname#1\endcsname{\csname orig#1\endcsname\csname #2\endcsname}%
}

\def\gaddjfont#1#2{{\globaldefs=1\addjfont{#1}{#2}}}

\def\gaddjfonts#1#2{{\globaldefs=1%
\cslet{orig#1fonts}{#1fonts}%
\expandafter\def\csname#1fonts\endcsname{\csname orig#1fonts\endcsname\cslet{tendm}{#2dm}\cslet{tendg}{#2dg}}%
}}

\gaddjfont{rm}{tendm}
\gaddjfont{bf}{tendg}
\gaddjfont{sl}{tendg}
\gaddjfont{authorrm}{secdm}

\gaddjfonts{text}{text}
\gaddjfonts{title}{title}
\gaddjfonts{chap}{chap}
\gaddjfonts{sec}{sec}
\gaddjfonts{subsec}{ssec}
\gaddjfonts{small}{small}

\global\let\subsubsecfonts = \subsecfonts
\global\let\subsecentryfonts = \textfonts
\global\let\subsubsecentryfonts = \textfonts

%%%%%%%%%%%%%%%%%%%%%%%%%%%%%%%%%%%%%%%%%%%%%%%%%%%%%%%%%%%%%%%%
%%
%% Utility routines.
%%

\def\gaddsequence#1{%
\if#1[%]
\def\next{\gaddsequencez#1}%
\else%
\def\next{\gaddsequencez[0]{#1}}%
\fi%
\next%
}

\def\gaddsequencez[#1]#2{%
\if#2[%]
\def\next{\gaddsequencezz[#1]#2}%
\else%
\def\next{\gaddsequencezz[#1][lb]{#2}}%
\fi%
\next%
}

\newif\ifL\newif\ifB
\def\aslb{\Ltrue\Btrue}
\def\asla{\Ltrue\Bfalse}
\def\asgb{\Lfalse\Btrue}
\def\asga{\Lfalse\Bfalse}

%%
%% 汚いマクロだなぁ〜
%%

\def\gaddsequencezz[#1][#2]#3#4{%
\global\cslet{orig#3}{#3}%
\csname as#2\endcsname%
\ifcase#1%
\ifL
 \ifB
  \expandafter\gdef\csname#3\endcsname{{#4\csname orig#3\endcsname}}%
 \else
  \expandafter\gdef\csname#3\endcsname{{\csname orig#3\endcsname#4}}%
 \fi
\else
 \ifB
  \expandafter\gdef\csname#3\endcsname{#3\csname orig#3\endcsname}%
 \else
  \expandafter\gdef\csname#3\endcsname{\csname orig#3\endcsname#3}%
 \fi
\fi
\or
\ifL
 \ifB
  \expandafter\gdef\csname#3\endcsname##1{{#4\csname orig#3\endcsname{##1}}}%
 \else
  \expandafter\gdef\csname#3\endcsname##1{{\csname orig#3\endcsname{##1}#4}}%
 \fi
\else
 \ifB
  \expandafter\gdef\csname#3\endcsname##1{#4\csname orig#3\endcsname{##1}}%
 \else
  \expandafter\gdef\csname#3\endcsname##1{\csname orig#3\endcsname{##1}#4}%
 \fi
\fi
\fi
}

%%%%%%%%%%%%%%%%%%%%%%%%%%%%%%%%%%%%%%%%%%%%%%%%%%%%%%%%%%%%%%%%%%

\gaddsequence[1]{initial}{%
\addjfont{secbf}{secdg}%
}

\gaddsequence{summarycontents}{%
\addjfont{shortcontrm}{shortcontdm}%
\addjfont{shortcontbf}{shortcontdg}%
\addjfont{shortcontsl}{shortcontdg}%
}
\global\let\shortcontents = \summarycontents

\gaddsequence{shorttitlepagezzz}{%
\addjfont{chaprm}{chapdm}%
}

%%
%%

\global\def\thischapterspace{\hskip \SETthischapterspace em}
\set thischapterspace 1

\gaddsequence[1][ga]{chapterzzz}{%
\xdef\thischapter{第\the\chapno\putwordChapter{}\thischapterspace\noexpand\thischaptername}%
}

\gaddsequence[1][ga]{appendixzzz}{%
\xdef\thischapter{\putwordAppendix{}\appendixletter\thischapterspace\noexpand\thischaptername}%
}

%%%%%%%%%%%%%%%%%%%%%%%%%%%%%%%%%%%%%%%%%%%%%%%%%%%%%%%%%%%%%%%%
%%
%%
%%

\global\def\inforefzzz #1,#2,#3,#4**{\putwordInfo{}\putwordfile{} \file{\ignorespaces #3{}},  ノード\samp{\ignorespaces#1{}}\putwordSee{}}

\global\def\pxref#1{\xrefX[#1,,,,,,,]\putwordsee{}}
\global\def\xref#1{\xrefX[#1,,,,,,,]\putwordSee{}}

\global\def\xrefX[#1,#2,#3,#4,#5,#6]{\begingroup
  \unsepspaces
  \def\printedmanual{\ignorespaces #5}%
  \def\printednodename{\ignorespaces #3}%
  \setbox1=\hbox{\printedmanual}%
  \setbox0=\hbox{\printednodename}%
  \ifdim \wd0 = 0pt
    % No printed node name was explicitly given.
    \expandafter\ifx\csname SETxref-automatic-section-title\endcsname\relax
      % Use the node name inside the square brackets.
      \def\printednodename{\ignorespaces #1}%
    \else
      % Use the actual chapter/section title appear inside
      % the square brackets.  Use the real section title if we have it.
      \ifdim \wd1 > 0pt
        % It is in another manual, so we don't have it.
        \def\printednodename{\ignorespaces #1}%
      \else
        \ifhavexrefs
          % We know the real title if we have the xref values.
          \def\printednodename{\refx{#1-title}{}}%
        \else
          % Otherwise just copy the Info node name.
          \def\printednodename{\ignorespaces #1}%
        \fi%
      \fi
    \fi
  \fi
  %
  % If we use \unhbox0 and \unhbox1 to print the node names, TeX does not
  % insert empty discretionaries after hyphens, which means that it will
  % not find a line break at a hyphen in a node names.  Since some manuals
  % are best written with fairly long node names, containing hyphens, this
  % is a loss.  Therefore, we give the text of the node name again, so it
  % is as if TeX is seeing it for the first time.
%  \ifpdf
%    \leavevmode
%    \getfilename{#4}%
%    \ifnum\filenamelength>0
%      \startlink attr{/Border [0 0 0]}%
%	 goto file{\the\filename.pdf} name{#1@}%
%    \else
%      \startlink attr{/Border [0 0 0]}%
%	 goto name{#1@}%
%    \fi
%    \linkcolor
%  \fi
  %
  \ifdim \wd1 > 0pt
%    \putwordsection{} ``\printednodename'' \putwordin{} \cite{\printedmanual}%
    \cite{\printedmanual}の``\printednodename''\putwordsection{}%
  \else
    % _ (for example) has to be the character _ for the purposes of the
    % control sequence corresponding to the node, but it has to expand
    % into the usual \leavevmode...\vrule stuff for purposes of
    % printing. So we \turnoffactive for the \refx-snt, back on for the
    % printing, back off for the \refx-pg.
    {\normalturnoffactive
     % Only output a following space if the -snt ref is nonempty; for
     % @unnumbered and @anchor, it won't be.
     \setbox2 = \hbox{\ignorespaces \refx{#1-snt}{}}%
%     \ifdim \wd2 > 0pt \refx{#1-snt}\space\fi
     \ifdim \wd2 > 0pt \refx{#1-snt}\fi
    }%
    % [mynode],
    「\printednodename 」%
%    [\printednodename],\space
    % page 3
    \turnoffactive \putwordpage\tie\refx{#1-pg}{}%
  \fi
  \endlink
\endgroup}

\global\def\Ysectionnumberandtype{%
\ifnum\secno=0 第\the\chapno\putwordChapter%
\else \ifnum \subsecno=0 \the\chapno.\the\secno\putwordSection%
\else \ifnum \subsubsecno=0 %
\the\chapno.\the\secno.\the\subsecno\putwordSection%
\else %
\the\chapno.\the\secno.\the\subsecno.\the\subsubsecno\putwordSection%
\fi \fi \fi }

\global\def\Yappendixletterandtype{%
\ifnum\secno=0 \putwordAppendix\xreftie'char\the\appendixno{}%
\else \ifnum \subsecno=0 \xreftie'char\the\appendixno.\the\secno\putwordSection %
\else \ifnum \subsubsecno=0 %
\xreftie'char\the\appendixno.\the\secno.\the\subsecno\putwordSection %
\else %
\xreftie'char\the\appendixno.\the\secno.\the\subsecno.\the\subsubsecno\putwordSection %
\fi \fi \fi }

%%%%%%%%%%%%%%%%%%%%%%%%%%%%%%%%%%%%%%%%%%%%%%%%%%%%%%%%%%%%%%%%

% @dfn
\global\def\doublebracket#1{『#1』}
\global\let\dfn=\doublebracket

%%%%%%%%%%%%%%%%%%%%%%%%%%%%%%%%%%%%%%%%%%%%%%%%%%%%%%%%%%%%%%%%
