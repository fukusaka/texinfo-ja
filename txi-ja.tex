%% TeX macros to handle Japanese texinfo files for Egg
%% Modified by Satoru Tomura (tomura@etl.go.jp)
%% 92.7.8   modified for Mule Ver.0.9.5 by K.Handa <handa@etl.go.jp>
%%      To detect type of jTeX and its version, the method
%%      posted by Takafumi SAKURAI <sakurai@math.metro-u.ac.jp> is used.
%% 92.9.30  modified for Mule Ver.0.9.6 by K.Handa <handa@etl.go.jp>
%%      For unknown reason, \newif\ifNTTOLD should be before
%%      \ifNTT.
%% 93.4.29  modified for Mule Ver.0.9.7 by N.Hikichi <hikichi@sra.co.jp>
%% 95.10.6  modified for texinfo 2.145 by K.Handa <handa@etl.go.jp>
%% 95.10.13 modified by J.Sato <jun@svgw.rd.casio.co.jp>
%%      Support many Japanese oriented phrases (reference, etc)
%% 95.10.14 modified by K.Handa <handa@etl.go.jp>
%%      Bug for handling index fixed.
%% 96.1.16 modified by J.Sato <jun@svgw.rd.casio.co.jp>
%%      index with [] of @deffn.
%% 99.6.27 modified by Moimoi <fukusaka@xa2.so-net.ne.jp>
%%	for texinfo.tex 1999-05-25.6
%% 2000.2.23 modified by Moimoi <fukusaka@xa2.so-net.ne.jp>
%%	for texinfo.tex 1999-09-25.10

%%%%%%%%%%%%%%%%%%%%%%%%%%%%%%%%%%%%%%%%%%%%%%%%%%%%%%%%%%%%%%%%
%% 92.7.8 by K.Handa
\newif\ifNTT
\ifx\gtfam\undefined
\NTTtrue
\else
\NTTfalse
\fi

\newif\ifNTTOLD
\ifNTT
\ifx\jendlinetype\undefined
\NTTOLDtrue
\else
\NTTOLDfalse
\fi
\fi
%% end of patch

%% TeX macros to handle Japanese texinfo files
%% 92/05/24 merged jtexinfo.tex (by H. Isozaki and N. Hikichi) into this
%% Created by Satoru Tomura (tomura@etl.go.jp)

\def\texinfoJPversion{2.145J.2+}

\ifNTT
\message{txi-ja (NTT JTeX) package [Version \texinfoJPversion]:}
\else
\message{txi-ja (ASCII JTeX) package [Version \texinfoJPversion]:}
\fi
\message{}

%%%%%%%%%%%%%%%%%%%%%%%%%%%%%%%%%%%%%%%%%%%%%%%%%%%%%%%%%%%%%%%%
% Set up fixed words for Japanese.
\gdef\putwordAppendix{付録}
\gdef\putwordChapter{章}
\gdef\putwordfile{ファイル}
\gdef\putwordin{in}
\gdef\putwordIndexIsEmpty{(索引が空です)}
\gdef\putwordIndexNonexistent{(索引がありません)}
\gdef\putwordInfo{Info}
\gdef\putwordInstanceVariableof{Instance Variable of}
\gdef\putwordMethodon{Method on}
\gdef\putwordNoTitle{無タイトル}
\gdef\putwordof{of}
\gdef\putwordon{on}
\gdef\putwordpage{p.\gobble}
\gdef\putwordsection{節}
\gdef\putwordSection{節}
\gdef\putwordsee{参照}
\gdef\putwordSee{を参照してください}
\gdef\putwordShortTOC{簡略目次}
\gdef\putwordTOC{目次}
%
\global\newif\ifIGUMO\IGUMOfalse
\gdef\igumo{\IGUMOtrue}
\gdef\putwordMJan{睦月}
\gdef\putwordMFeb{如月}
\gdef\putwordMMar{弥生}
\gdef\putwordMApr{卯月}
\gdef\putwordMMay{皐月}
\gdef\putwordMJun{水無月}
\gdef\putwordMJul{文月}
\gdef\putwordMAug{葉月}
\gdef\putwordMSep{長月}
\gdef\putwordMOct{\ifIGUMO{神在月}\else{神無月}\fi}
\gdef\putwordMNov{霜月}
\gdef\putwordMDec{師走}
%
\gdef\putwordDefmac{マクロ}
\gdef\putwordDefspec{Special Form}
\gdef\putwordDefvar{変数}
\gdef\putwordDefopt{オプション}
\gdef\putwordDeftypevar{変数}
\gdef\putwordDeffunc{関数}
\gdef\putwordDeftypefun{関数}

\def\today{\number\year 年 \number\month 月 \number\day 日}

%%%%%%%%%%%%%%%%%%%%%%%%%%%%%%%%%%%%%%%%%%%%%%%%%%%%%%%%%%%%%%%%
%
% A4 size(Japanese) define, top margin = 20, bottom margin = 21,
%  left margin = 30, right margin = 15
%

% ???
%\global\def\a4book{
%\global\lispnarrowing = 0.3in
%\global\baselineskip 12pt
%\global\parskip 3pt plus 1pt
%
%% for @cropmarks
%%\global\hsize = 6.5in
%% without @cropmarks
%\global\hsize = 6.7in
%
%\global\doublecolumnhsize=2.4in \global\doublecolumnvsize=15.0in
%\global\vsize=9.8in
%\global\tolerance=700
%\global\hfuzz=1pt
%
%\global\pagewidth=\hsize
%\global\pageheight=\vsize
%\global\font\ninett=cmtt9
%
%\global\let\smalllisp=\smalllispx
%\global\let\smallexample=\smalllispx
%\global\def\Esmallexample{\Esmalllisp}
%
%% for @cropmarks
%%\global\voffset = -1.0in
%%\global\hoffset = -0.2in
%
%% without @cropmarks
%\global\voffset = 0.0in
%%\global\hoffset = -1.0in
%\global\hoffset = -0.2in
%}

% hack please !!
\global\def\afourbook{{\globaldefs = 1
  \setleading{12pt}%
  \parskip = 3pt plus 2pt minus 1pt
  %
  \internalpagesizes{248mm}{170mm}{0mm}{-5mm}{0mm}{8mm}%
  %
  \tolerance = 700
  \hfuzz = 1pt
}}

%% @smallbook for B5
%\global\def\smallbook{
%\outerhsize=182mm
%\outervsize=257mm
%\hoffset=-0.3in
%\voffset=-0.3in
%
%% These values for secheadingskip and subsecheadingskip are
%% experiments.  RJC 7 Aug 1992
%\global\secheadingskip = 17pt plus 6pt minus 3pt
%\global\subsecheadingskip = 14pt plus 6pt minus 3pt
%
%\global\lispnarrowing = 0.3in
%\setleading{14pt}
%\advance\topskip by -7mm
%\global\parskip 3pt plus 1pt
%\global\hsize = 5.5in
%\global\vsize=8.25in
%\global\tolerance=700
%\global\hfuzz=1pt
%\global\contentsrightmargin=0pt
%\global\deftypemargin=0pt
%\global\defbodyindent=.5cm
%
%\global\pagewidth=\hsize
%\global\pageheight=\vsize
%
%\global\let\smalllisp=\smalllispx
%\global\let\smallexample=\smalllispx
%\global\def\Esmallexample{\Esmalllisp}
%}

%%%%%%%%%%%%%%%%%%%%%%%%%%%%%%%%%%%%%%%%%%%%%%%%%%%%%%%%%%%%%%%%
%% 日本語フォントに関する互換性
\ifNTT
\global\kanjifiletype=20
\global\let\min=\dm\global\let\dg=\goth
\else
\global\let\dm=\min\global\let\goth=\dg
\fi

\global\setfont\textrm\rmshape{10}{1000}
\global\setfont\texttt\ttshape{10}{1000}
\global\setfont\textbf\bfshape{10}{1000}
\global\setfont\textit\itshape{10}{1000}
\global\setfont\textsl\slshape{10}{1000}
\global\setfont\textsf\sfshape{10}{1000}
\global\setfont\textsc\scshape{10}{1000}
\global\font\texti=cmmi10

%% 日本語フォントの定義
%% Fonts for text
\ifNTT
\ifNTTOLD                       % 92.7.8 by K.Handa
\global\jfont\textdm=dm10 scaled {\magstephalf}
\global\jfont\textdg=dg10 scaled {\magstephalf}
\global\jfont\shortcontdm=dm10 scaled {\magstep1}
\global\jfont\shortcontdg=dg10 scaled {\magsteph1}
\else
%\jfont\textdm=dm10 scaled \magstephalf
%\jfont\textdg=dg10 scaled \magstephalf
%\jfont\shortcontdm=dm10 scaled \magstephalf
\global\jfont\textdm=dm10
\global\jfont\textdg=dg10
\global\jfont\shortcontdm=dm12
\global\jfont\shortcontdg=dg12
\fi
\else
%\font\textdm=min10 scaled \magstephalf
%\font\textdg=goth10 scaled \magstephalf
%\jfont\shortcontdm=min10 scaled \magstephalf
\global\font\textdm=min10
\global\font\textdg=goth10
\global\jfont\shortcontdm=min10 scaled \magstep1
\global\jfont\shortcontdg=goth10 scaled \magstep1
\fi

%% Fonts for title
\ifNTT
\global\jfont\titledm=dg12 scaled \magstep3
\else
%\font\titledm=goth12 scaled \magstep3
\global\font\titledm=goth10 scaled \magstep4
\fi

%% Fonts for indics and small examples
\ifNTT
\global\jfont\smalldm=dm9
\global\jfont\smalldg=dm9
\else
\global\font\smalldm=min9
\global\font\smalldg=min9
\fi

%% Fonts for headings
\ifNTT
\global\jfont\chapdm=dg12 scaled \magstep2
\else
%\font\chapdm=min12 scaled \magstep2
\global\font\chapdm=goth10 scaled \magstep3
\fi
\global\let\chapdg=\chapdm

\ifNTT
\global\jfont\secdm=dg12 scaled \magstep1
\global\jfont\secdg=dg12 scaled \magstep1
\else
%\font\secdm=min12 scaled \magstep1
\global\font\secdm=goth10 scaled \magstep2
%\font\secdg=goth12 scaled \magstep1
\global\font\secdg=goth10 scaled \magstep2
\fi

\ifNTT
\ifNTTOLD                       % 92.7.8 by K.Handa
\global\jfont\ssecdm=dg12 scaled {\magstephalf}
\global\jfont\ssecdg=dg12 scaled {\magstephalf}
\else
\global\jfont\ssecdm=dg12 scaled \magstephalf
\global\jfont\ssecdg=dg12 scaled \magstephalf
\fi
\else
%\font\ssecdm=goth12 scaled \magstephalf
\global\font\ssecdm=goth10 at 13pt
%\font\ssecdg=goth12 scaled \magstephalf
\global\font\ssecdg=goth10 at 13pt
\fi

% 95.11.2 by K.Handa
% Reduce Overfull/Underfull \hbox by relaxing these glues.
\ifNTT
\global\jintercharskip=0pt plus 0.5pt minus -0.2pt
\global\jasciikanjiskip=2.28854pt plus 0.5pt minus -0.2pt
\fi

%% (^^;)
\global\def\tendm{}
\global\def\tendg{}

%% Re-definitions
\def\addjfont#1#2{
\global\cslet{orig#1}{#1}
\global\expandafter\def\csname#1\endcsname{\csname orig#1\endcsname\csname #2\endcsname}}

\def\addjfonts#1#2{
\global\cslet{orig#1fonts}{#1fonts}
\global\expandafter\def\csname#1fonts\endcsname
{\cslet{tendm}{#2dm}\cslet{tendg}{#2dg}\csname orig#1fonts\endcsname}}

\addjfont{rm}{tendm}
\addjfont{bf}{tendg}
\addjfont{sl}{tendg}
\addjfont{authorrm}{secdm}

\addjfonts{text}{text}
\addjfonts{title}{title}
\addjfonts{chap}{chap}
\addjfonts{sec}{sec}
\addjfonts{subsec}{ssec}
\addjfonts{small}{small}

\global\let\subsubsecfonts=\subsecfonts
\global\let\subsecentryfonts = \textfonts
\global\let\subsubsecentryfonts = \textfonts

%%%%%%%%%%%%%%%%%%%%%%%%%%%%%%%%%%%%%%%%%%%%%%%%%%%%%%%%%%%%%%%%

\global\def\initial#1{{%
  % Some minor font changes for the special characters.
  %\let\tentt=\sectt \let\tt=\sectt \let\sf=\sectt
  %
  % Remove any glue we may have, we'll be inserting our own.
  \removelastskip
  %
  % We like breaks before the index initials, so insert a bonus.
  \penalty -300
  %
  % Typeset the initial.  Making this add up to a whole number of
  % baselineskips increases the chance of the dots lining up from column
  % to column.  It still won't often be perfect, because of the stretch
  % we need before each entry, but it's better.
  %
  % No shrink because it confuses \balancecolumns.
  \vskip 1.67\baselineskip plus .5\baselineskip
  %\leftline{\secbf\secdg #1}%
  \leftline{\secfonts\bf #1}%
  \vskip .33\baselineskip plus .1\baselineskip
  %
  % Do our best not to break after the initial.
  \nobreak
}}

% \shortcontfonts ...
\global\def\shortchaplabel#1{%
  % Compute width of word "Appendix", may change with language.
  \setbox0 = \hbox{\shortcontrm\shortcontdm \putwordAppendix}%
  \shortappendixwidth = \wd0
  %
  % We typeset #1 in a box of constant width, regardless of the text of
  % #1, so the chapter titles will come out aligned.
  \setbox0 = \hbox{#1}%
  \dimen0 = \ifdim\wd0 > \shortappendixwidth \shortappendixwidth \else 0pt \fi
  %
  % This space should be plenty, since a single number is .5em, and the
  % widest letter (M) is 1em, at least in the Computer Modern fonts.
  % (This space doesn't include the extra space that gets added after
  % the label; that gets put in by \shortchapentry above.)
  \advance\dimen0 by 1.1em
  \hbox to \dimen0{#1\hfil}%
}

% And just the chapters.
\global\def\summarycontents{%
   \startcontents{\putwordShortTOC}%
      %
      \let\chapentry = \shortchapentry
      \let\unnumbchapentry = \shortunnumberedentry
      % We want a true roman here for the page numbers.
      \secfonts
      \def\rm{\shortcontrm\shortcontdm}
      \def\bf{\shortcontbf\shortcontdg}
      \def\sl{\shortcontsl\shortcontdg}
      \rm
      \hyphenpenalty = 10000
      \advance\baselineskip by 1pt % Open it up a little.
      \def\secentry ##1##2##3##4{}
      \def\unnumbsecentry ##1##2{}
      \def\subsecentry ##1##2##3##4##5{}
      \def\unnumbsubsecentry ##1##2{}
      \def\subsubsecentry ##1##2##3##4##5##6{}
      \def\unnumbsubsubsecentry ##1##2{}
      \openin 1 \jobname.toc
      \ifeof 1 \else
        \closein 1
        \input \jobname.toc
      \fi
     \vfill \eject
     \contentsalignmacro % in case @setchapternewpage odd is in effect
   \endgroup
   \lastnegativepageno = \pageno
   \pageno = \savepageno
}
\global\let\shortcontents = \summarycontents

\global\def\shorttitlepagezzz #1{\begingroup\hbox{}\vskip 1.5in \chaprm\chapdm \centerline{#1}%
        \endgroup\page\hbox{}\page}


%%%%%%%%%%%%%%%%%%%%%%%%%%%%%%%%%%%%%%%%%%%%%%%%%%%%%%%%%%%%%%%%

\global\def\thischapterspace{\hskip 1em}

\global\let\origchapterzzz = \chapterzzz
\global\def\chapterzzz #1{%
\origchapterzzz{#1}%
\xdef\thischapter{第\the\chapno\putwordChapter{}\thischapterspace\noexpand\thischaptername}%
}

\global\let\origappendixzzz = \appendixzzz
\global\def\appendixzzz #1{%
\origappendixzzz{#1}%
\xdef\thischapter{\putwordAppendix{}\appendixletter\thischapterspace\noexpand\thischaptername}%
}

\global\def\inforefzzz #1,#2,#3,#4**{\putwordInfo{}\putwordfile{} \file{\ignorespaces #3{}},  ノード\samp{\ignorespaces#1{}}\putwordSee{}}

\global\def\pxref#1{\xrefX[#1,,,,,,,]\putwordsee{}}
\global\def\xref#1{\xrefX[#1,,,,,,,]\putwordSee{}}

\global\def\xrefX[#1,#2,#3,#4,#5,#6]{\begingroup
  \unsepspaces
  \def\printedmanual{\ignorespaces #5}%
  \def\printednodename{\ignorespaces #3}%
  \setbox1=\hbox{\printedmanual}%
  \setbox0=\hbox{\printednodename}%
  \ifdim \wd0 = 0pt
    % No printed node name was explicitly given.
    \expandafter\ifx\csname SETxref-automatic-section-title\endcsname\relax
      % Use the node name inside the square brackets.
      \def\printednodename{\ignorespaces #1}%
    \else
      % Use the actual chapter/section title appear inside
      % the square brackets.  Use the real section title if we have it.
      \ifdim \wd1 > 0pt
        % It is in another manual, so we don't have it.
        \def\printednodename{\ignorespaces #1}%
      \else
        \ifhavexrefs
          % We know the real title if we have the xref values.
          \def\printednodename{\refx{#1-title}{}}%
        \else
          % Otherwise just copy the Info node name.
          \def\printednodename{\ignorespaces #1}%
        \fi%
      \fi
    \fi
  \fi
  %
  % If we use \unhbox0 and \unhbox1 to print the node names, TeX does not
  % insert empty discretionaries after hyphens, which means that it will
  % not find a line break at a hyphen in a node names.  Since some manuals
  % are best written with fairly long node names, containing hyphens, this
  % is a loss.  Therefore, we give the text of the node name again, so it
  % is as if TeX is seeing it for the first time.
%  \ifpdf
%    \leavevmode
%    \getfilename{#4}%
%    \ifnum\filenamelength>0
%      \startlink attr{/Border [0 0 0]}%
%	 goto file{\the\filename.pdf} name{#1@}%
%    \else
%      \startlink attr{/Border [0 0 0]}%
%	 goto name{#1@}%
%    \fi
%    \linkcolor
%  \fi
  %
  \ifdim \wd1 > 0pt
%    \putwordsection{} ``\printednodename'' \putwordin{} \cite{\printedmanual}%
    \cite{\printedmanual}の``\printednodename''\putwordsection{}%
  \else
    % _ (for example) has to be the character _ for the purposes of the
    % control sequence corresponding to the node, but it has to expand
    % into the usual \leavevmode...\vrule stuff for purposes of
    % printing. So we \turnoffactive for the \refx-snt, back on for the
    % printing, back off for the \refx-pg.
    {\normalturnoffactive
     % Only output a following space if the -snt ref is nonempty; for
     % @unnumbered and @anchor, it won't be.
     \setbox2 = \hbox{\ignorespaces \refx{#1-snt}{}}%
%     \ifdim \wd2 > 0pt \refx{#1-snt}\space\fi
     \ifdim \wd2 > 0pt \refx{#1-snt}\fi
    }%
    % [mynode],
    「\printednodename 」%
%    [\printednodename],\space
    % page 3
    \turnoffactive \putwordpage\tie\refx{#1-pg}{}%
  \fi
  \endlink
\endgroup}

\global\def\Ysectionnumberandtype{%
\ifnum\secno=0 第\the\chapno\putwordChapter%
\else \ifnum \subsecno=0 \the\chapno.\the\secno\putwordSection%
\else \ifnum \subsubsecno=0 %
\the\chapno.\the\secno.\the\subsecno\putwordSection%
\else %
\the\chapno.\the\secno.\the\subsecno.\the\subsubsecno\putwordSection%
\fi \fi \fi }

\global\def\Yappendixletterandtype{%
\ifnum\secno=0 \putwordAppendix\xreftie'char\the\appendixno{}%
\else \ifnum \subsecno=0 \xreftie'char\the\appendixno.\the\secno\putwordSection %
\else \ifnum \subsubsecno=0 %
\xreftie'char\the\appendixno.\the\secno.\the\subsecno\putwordSection %
\else %
\xreftie'char\the\appendixno.\the\secno.\the\subsecno.\the\subsubsecno\putwordSection %
\fi \fi \fi }

%%%%%%%%%%%%%%%%%%%%%%%%%%%%%%%%%%%%%%%%%%%%%%%%%%%%%%%%%%%%%%%%

% @dfn
\global\def\doublebracket#1{『#1』}
\global\let\dfn=\doublebracket

%%%%%%%%%%%%%%%%%%%%%%%%%%%%%%%%%%%%%%%%%%%%%%%%%%%%%%%%%%%%%%%%
